\begin{table}[htbp]\centering
\small
\caption{Regression Discontinuity Estimates for the Effect of Exceeding BAC Thresholds on Predetermined Driver Characteristics}
\begin{center}
\begin{threeparttable}
\begin{tabular}{l*{4}{c}}
\toprule
\multicolumn{1}{l}{\textit{Dependent var: }}&
\multicolumn{1}{c}{\textit{Male}}&
\multicolumn{1}{c}{\textit{White}}&
\multicolumn{1}{c}{\textit{Age}}&
\multicolumn{1}{c}{\textit{Accident}}\\
\midrule
dui                 &      -0.001   &       0.022   &       0.556   &       0.005   \\
                    &     (0.019)   &     (0.016)   &     (0.532)   &     (0.013)   \\
bac1\_c              &      -3.963   &      -1.048   &     -81.657   &      -0.721   \\
                    &     (3.261)   &     (2.864)   &    (94.213)   &     (2.278)   \\
interaction         &       4.280   &      -4.212   &    -122.804   &      -1.338   \\
                    &     (6.731)   &     (5.630)   &   (190.447)   &     (4.503)   \\
\midrule
N                   &       9,514   &       9,514   &       9,514   &       9,514   \\
Mean of dependent variable&        0.79   &        0.85   &       34.11   &        0.09   \\
\bottomrule
\end{tabular}
\begin{tablenotes}
\tiny
\item This table contains regression discontinuity based estimates of the effect of having BAC above the legal thresholds on predetermined  drivers characteristics based on Hansen (2015).  Heteroskedastic standard errors shown in parenthesis.  * p$<$0.10, ** p$<$0.05, *** p$<$0.01
\end{tablenotes}
\end{threeparttable}
\end{center}
\end{table}
